\section{Теория}
\subsection{Функции распределения}
\begin{itemize}
    \item Нормальное распределение 
    \begin{equation}
        N(x, 0, 1) = \frac{1}{\sqrt{2\pi}}e^\frac{-x^2}{2}
        \label{eq1}
    \end{equation}
    
    \item Распределение Коши 
    \begin{equation}
       C(x, 0, 1) = \frac{1}{\pi}\frac{1}{x^2+1}
       \label{eq2}
    \end{equation}
    
    \item Распределение Стьюдента $t(x, 0, 3)$ с тремя степенями свободы
     \begin{equation}
       t(x, 0, 3) = \frac{6\sqrt3}{\pi(3 + t^2)^2}
       \label{eq3}
    \end{equation}
    
    \item Распределение Пуассона 
     \begin{equation}
       P(k, 10) = \frac{10^k}{k!}e^{-10}
       \label{eq4}
    \end{equation}
    
    \item Равномерное распределение 
     \begin{equation}
       U(x, -\sqrt3, \sqrt3) = \begin{cases}
       \frac{1}{2\sqrt3} & \mbox{при} \; |x| \leq \sqrt3\\
       0 & \mbox{при} \; |x| > \sqrt3
       \end{cases}
       \label{eq5}
    \end{equation}
    
\end{itemize}

\subsection{Характеристики положения}

\begin{itemize}
    \item Выборочное среднее
    \begin{equation}
        \overline{x} = \tfrac{1}{n}\sum_{i = 1}^{n}x_i
        \label{eq6}
    \end{equation}
    \item Выборочная медиана
    
       \begin{equation}
med\ x = \left\{
\begin{array}{ccl}
x_{(l + 1)} & \text{при} & n = 2l + 1\\
\dfrac{x_{(l)} + x_{(l + 1)}}{2} & \text{при} & n = 2l
\end{array}
\right.
\label{eq7}
\end{equation}
    
    \item Полусумма экстремальных выборочных элементов
    \begin{equation}
        z_{R} = \frac{x_{(1)} + x_{(n)}}{2}
        \label{eq8}
    \end{equation}
    \item Полусумма квартилей\\
    Выборочная квартиль $z_{p}$ порядка $p$ определяется формулой
    \begin{equation}
        z_{p} = \left\{
\begin{array}{ccl}
x_{([np]+ 1)} & \text{при} & np\ \text{дробном}\\
x_{(np)} & \text{при} & np\ \text{целом}
\end{array}
\right.
\label{eq9}
    \end{equation}
    Полусумма квартилей\\
    \begin{equation}
        z_{Q} = \dfrac{z_{1/4} + z_{3/4}}{2}
        \label{eq10}
    \end{equation}
    \item Усечённое среднее
    
    \begin{equation}
        z_{tr} = \tfrac{1}{n - 2r}\sum_{i = r + 1}^{n - r}x_{(i)},\ r\approx\dfrac{n}{4}
        \label{eq11}
    \end{equation}
    \item Оценка дисперсии
    \begin{equation}
        D(z) = \overline{z^2} - \overline{z}^2
        \label{eq12}
    \end{equation}
\end{itemize}

\subsection{Бокс-плот Тьюки}
Боксплот (англ. box plot) — график, использующийся в описательной статистике, компактно изображающий одномерное распределение вероятностей.
Такой вид диаграммы в удобной форме показывает медиану, нижний и верхний квартили и выбросы. Границами ящика служат первый и третий квартили, линия в середине ящика — медиана. Концы усов — края статистически
значимой выборки (без выбросов). Длину «усов» определяют разность первого квартиля и полутора межквартильных расстояний и сумма третьего
квартиля и полутора межквартильных расстояний. Формула имеет вид

\begin{equation}
    X_{1} = Q_{1} - \dfrac{3}{2}(Q_{3} -Q_{1}),\;\; X_{2} = Q_{3} + \dfrac{3}{2}(Q_{3} -Q_{1})
    \label{eq13}
\end{equation}

\noindentгде $X_{1}$ — нижняя граница уса, $X_{2}$ — верхняя граница уса, $Q_{1}$ — первый
квартиль, $Q_{3}$ — третий квартиль. Данные, выходящие за границы усов (выбросы), отображаются на графике в виде маленьких кружков.\\
Выбросами считаются величины $x$, такие что


\begin{equation}
    \left[
  \begin{array}{ccc}
        x < X_{1}^{T}\\
        x > X_{2}^{T}
  \end{array}
\right.
\label{eq14}
\end{equation}

\subsection{ Доверительные интервалы для параметров нормального распределения}

\subsubsection{Доверительный интервал для среднего значения m нормального распределения}

Дана выборка $(x_1, x_2, ... x_n)$ объёма $n$ из нормальной генеральной совокупности. На её основе строим выборочное среднее $\overline{x}$ и выборочное среднее квадратическоеп отклонение $s$. Параметры $m$ и $\sigma$ нормального распределения неизвестны.\\
Доказано, что случайная величина:
\begin{equation}
    T = \sqrt{n - 1}\cdot \dfrac{\overline{x} - m}{s}
    \label{eq15}
\end{equation}

\noindent называемая статистикой Стьюдента, распределена по закону Стьюдента с $n-1$ степенями свободы. Пусть $f_{T}(x)$ - плотность вероятности этого распределения. Тогда 

\begin{multline*}
    P(-x < \sqrt{n - 1}\cdot \dfrac{\overline{x} - m}{s} < x) = P(-x < \sqrt{n - 1}\cdot \dfrac{m - \overline{x}}{s} < x) = \\ =\int \limits_{-x}^{x}f_{T}(t)dt = 2\int \limits_{0}^{x}f_{T}(t)dt = 2\Bigg(\int \limits_{-\infty}^{x}f_{T}(t)dt - \dfrac{1}{2}\Bigg) = 2F_{T}(x) - 1
\end{multline*}

\noindent Здесь  $F_{T}(x)$ - функция распределения Стьюдента с $n-1$ степенями свободы.\\
Полагаем $2F_{T}(x) - 1 = 1 - \alpha$, где $\alpha$ - выбранный уровень значимости. Тогда $F_{T}(x) = 1 - \sfrac{\alpha}{2}$. Пусть $t_{1 - \sfrac{\alpha}{2}}(n-1)$ - квантиль распределения Стьюдента с $n-1$ степенями свободы и порядка $1 - \sfrac{\alpha}{2}$. Из предыдущих равенств получаем

\begin{multline}
    P(\overline{x} - \dfrac{sx}{\sqrt{n - 1}} < m < \overline{x} + \dfrac{sx}{\sqrt{n - 1}}) = 2F_{T}(x) - 1 = 1 - \alpha, \\
    P(\overline{x} - \dfrac{st_{1 - \sfrac{\alpha}{2}}(n-1)}{\sqrt{n - 1}} < m < \overline{x} + \dfrac{st_{1 - \sfrac{\alpha}{2}}(n-1)}{\sqrt{n - 1}}) = 1 - \alpha
    \label{eq16}
\end{multline}

 \noindent что и даёт доверительный интервал для $m$ с доверительной верятностью $\gamma = 1 - \alpha$. 
        






\subsubsection{Доверительный интервал для среднего квалратического отклонения нормального распределения}

Дана выборка $(x_1, x_2, ... x_n)$ объёма $n$ из нормальной генеральной совокупности. На её основе строим выборочную дисперсию $s^2$.Параметры $m$ и $\sigma$ нормального распределения неизвестны.\\
Доказано, что случайная величина $\sfrac{ns^2}{\sigma^2}$ распределена по закону $\chi^2$ с $n-1$ степенями свободы.\\
Задаёмся уровнем значимости $\alpha$ и по таблице находим квантили $\chi_{\sfrac{\alpha}{2}}^{2}(n-1)$ и $\chi_{1 - \sfrac{\alpha}{2}}^{2}(n-1)$. Это значит, что\\
\begin{center}
    $P(\chi^{2}(n - 1) < \chi_{\sfrac{\alpha}{2}}^{2}(n-1)) = \sfrac{\alpha}{2}\;\;\;\;
    P(\chi^{2}(n - 1) < \chi_{1 - \sfrac{\alpha}{2}}^{2}(n-1)) = 1 - \sfrac{\alpha}{2}$
\end{center}
\noindent Откуда можно получить, что 
\begin{equation}
    P\Bigg(\dfrac{s\sqrt{n}}{\sqrt{\chi_{1 - \sfrac{\alpha}{2}}^{2}(n-1)}} < \sigma < \dfrac{s\sqrt{n}}{\sqrt{\chi_{\sfrac{\alpha}{2}}^{2}(n-1)}}\Bigg) = 1 - \alpha
    \label{eq17}
\end{equation}

\noindent что и даёт доверительный интервал для $\sigma$ с с доверительной верятностью $\gamma = 1 - \alpha$. 



\subsubsection{Доверительный интервал для среднего значения m произвольной генеральной совокупности при большом объёме выборки}

Пусть 
\begin{equation*}
    \Phi(x) = \dfrac{1}{\sqrt{2\pi}} \int \limits_{-\infty}
 ^{x}e^{\sfrac{-t^2}{2}}dt
 \end{equation*}
\noindent - функция Лапласа. тогда
$$P\Bigg(-x < \sqrt{n}\cdot \dfrac{\overline{x} - m}{\sigma} < x\Bigg) \approx 2\Phi(x) - 1$$
\noindent Откуда 
$$P\Bigg(\overline{x} - \dfrac{\sigma x}{\sqrt{n}} < m < \overline{x} + \dfrac{\sigma x}{\sqrt{n}}\Bigg) \approx 2\Phi(x) - 1$$

\noindent Положим $2\Phi(x) - 1 = \gamma = 1 -\alpha$. Пусть $u_{1 - \sfrac{\alpha}{2}}$ - квантиль нормального распределения $N(0, 1)$ порядка $1 - \sfrac{\alpha}{2}$. Получим

\begin{equation}
    P\Bigg(\overline{x} - \dfrac{su_{1 - \sfrac{\alpha}{2}}}{\sqrt{n}} < m < \overline{x} + \dfrac{su_{1 - \sfrac{\alpha}{2}}}{\sqrt{n}}\Bigg) \approx \gamma
    \label{eq18}
\end{equation}

\noindent что и даёт доверительный интервал для $m$ с с доверительной верятностью $\gamma = 1 - \alpha$. 


\subsubsection{Доверительный интервал для среднего квадратического отклонения произвольной генеральной совокупности при большом объёме выборки}

Пусть 
\begin{equation*}
    \Phi(x) = \dfrac{1}{\sqrt{2\pi}} \int \limits_{-\infty}
 ^{x}e^{\sfrac{-t^2}{2}}dt
 \end{equation*}
\noindent - функция Лапласа. Тогда 
$$P\Bigg(-x < \dfrac{s^2 - Ms^2}{\sqrt{Ds^2}} < x\Bigg) \approx 2\Phi(x) - 1$$

\noindent Положим $2\Phi(x) - 1 = \gamma = 1 -\alpha$. Пусть $u_{1 - \sfrac{\alpha}{2}}$ - квантиль нормального распределения $N(0, 1)$ порядка $1 - \sfrac{\alpha}{2}$. Получим

$$\sqrt{Ds^2} \approx \dfrac{\sigma^2}{\sqrt{n}}\sqrt{e + 2} $$

\begin{equation}
    s(1 + u_{1 - \sfrac{\alpha}{2}}\sqrt{\sfrac{(e+2)}{n}})^{\sfrac{-1}{2}} < \sigma < s(1 - u_{1 - \sfrac{\alpha}{2}}\sqrt{\sfrac{(e+2)}{n}})^{\sfrac{-1}{2}}
    \label{eq19}
\end{equation}
\noindent что и даёт доверительный интервал для $\sigma$ с с доверительной верятностью $\gamma = 1 - \alpha$. 