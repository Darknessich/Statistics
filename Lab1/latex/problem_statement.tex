\newpage
\section{Постановка задачи}

\subsection{Часть 1}
Для 5 распределений:
\begin{itemize}
    \item Нормальное распределение $N(x, 0, 1)$
    \item распределение Коши $C(x, 0, 1)$
    \item Распределение Стьюдента $t(x, 0, 3)$ с тремя степенями свободы
    \item Распределение Пуассона $P(k, 10)$
    \item Равномерное распределение $U(x, -\sqrt3, \sqrt3)$
\end{itemize}
Сгенерировать выборки размером 10, 50, 1000 элементов.\\
Построить на одном рисунке гистограмму и график плотности распределения.

\subsection{Часть 2}
Сгенерировать выборки размером 10, 100, 1000 элементов.\\
Для каждой выборки вычислить следующие статистические характеристики положения данных: $\overline{x}$, $med\:x$, $z_{Q}$, $z_{R}$, $z_{tr}$. Повторить такие вычисления 1000 раз для каждой выборки и найти среднее характеристик положения и их квадратов: $E(z) = \bar{z}$. Вычислить оценку дисперсии по формуле $D(z) = \overline{z^2} - \overline{z}^2$.

\subsection{Часть 3}
Для данных распределений сгенерировать выборки размером 20 и 100 элементов. Построить для них боксплот Тьюки.

\subsection{Часть 4}
Для данных распределений сгенерировать выборки размером 20 и 100 элементов. Вычислить параметры положения и рассеяния. Представить результаты графически.

