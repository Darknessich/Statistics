\newpage
\section{Обсуждение}
В ходе выполнения лабораторной работы (части 1 - 4) было изучено 5 видов распределений:
\begin{itemize}
    \item[-] Нормальное (\ref{eq1}),
    \item[-] Коши (\ref{eq2}),
    \item[-] Стьюдента (\ref{eq3}),
    \item[-] Пуассона (\ref{eq4}),
    \item[-] Равномерное (\ref{eq5}).
\end{itemize}

В 1-ой части работы были построены гистограммы для каждого распределения на основании выборок по 10, 50 и 1000 элементов. Дополнительно на тех же рисунках были построены графики соответствующих распределений. Это было необходимо для сравнения форм распределения выборок с теоретическими моделями.\\

Во 2-ой части работы были рассмотрены следующие характеристики: выборочное среднее (\ref{eq6}), выборочная медиана (\ref{eq7}), полусумма экстремальных выборочных элементов (\ref{eq8}), полусумма квартилей (\ref{eq10}), усечённое среднее (\ref{eq11}), дисперсия (\ref{eq12}).\\
По полученным результатам можно сделать вывод о том, что чем больше выборка для выбранного распределения, тем точнее гистограмма приближает график плотности распределения вероятностей. Соответственно, чем меньше выборка, тем менее точно она позволяет судить о характере распределения величины.\\
Для нормального распределения (рис.\ref{fig1}, таб.\ref{table:1}) увеличение размера выборки приводит к тому, что оценки характеристик положения и рассеяния приближаются к их теоретическим значениям.\\
Для распределение Коши (рис.\ref{fig2}, таб.\ref{table:2}) оказалась характерна чувствительность к выбросам. Этим объясняются большие значения дисперсии характеристик рассеяния и нестабильность среднего значения. В этом случае медиана менее подвержена влиянию выбросов, т.к. она не зависит от значений в хвостах распределения. Выборочные квартили и полусумма экстремальных выборочных значений также могут быть неустойчивыми, поскольку распределение Коши не имеет конечного математического ожидания и дисперсии.\\
При малом размере выборки оценки для распределения Стьюдента (рис.\ref{fig3}, таб.\ref{table:3}) также остаются неустойчивыми, но с увеличением выборки их точность растёт.\\
Для распределения Пуассона (рис.\ref{fig4}, таб.\ref{table:4}) медиана и среднее значение оказались практически схожими, что можно объяснить тем, что среднее значение в данном распределении равно его параметру $\lambda$, и эти показатели становятся близкими с увеличением выборки. Таким образом, ля распределения Пуассона и равномерного распределения (рис.\ref{fig5}, таб.\ref{table:5}) оценки характеристик положения и рассеяния остаются стабильными при любом размере выборки.\\

В ходе выполнения частей 3 и 4 лабораторной работы для 5-ти заданных распределений ((\ref{eq1}) - (\ref{eq5})) были сгенерированы выборки по 20 и 100 элементов. На их основе были построены бокс-плоты Тьюки (рис. \ref{fig6} - \ref{fig10}) и вычислены доверительные интервалы для параметров $m$ (среднее значение) и $\sigma$ (среднеквадратическое отклонение) нормального распределения и произвольного распределения (таб. \ref{table:6}, \ref{table:7}). Последние результаты были также представлены графически (рис. \ref{fig11}, \ref{fig12}).\\
На основании проведённого исследования можно сказать, что бокс-плоты Тьюки действительно позволяют оценивать значимые характеристики распределений более наглядно и с меньшими усилиями. Используя доверительные интервалы, можно оценить неопределенность среднего значения и стандартного отклонения параметров нормального и произвольного распределения. С увеличением размера выборки доверительный интервал уменьшается, а значит, оценить параметры можно более точно.\\
Вероятность выбросов в данных увеличивается с ростом объёма выборки. Для распределений с бесконечным носителем выбросы преимущественно можно наблюдать на бокс-плотах Тьюки для выборки размером 100 элементов ( рис. \ref{fig6} - \ref{fig9}). Равномерное распределение имеет конечный носитель, поэтому на рисунке \ref{fig10} выбросов нет.

